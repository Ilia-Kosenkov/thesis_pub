\chapter{Future research}
Accreting stellar-mass black holes show signs of variable intrinsic optical/near-infrared polarization both during outbursts and in quiescence.
The amplitude of polarization variability can be very small (of the order of 0.1\%), which requires exceptionally good polarimeters with accuracy down to 10$^{-5}$ in order to detect such subtle changes.
Instruments such as \DP\ and \DUF\ are excellent examples of polarimeters capable of achieving this.
Expanding the network of DIPol polarimeters, allocating more time for observations of transient sources, and creating a monitoring system which regularly samples polarization of a large number of X-ray binaries will allow us to generate more data and find similarities in the polarization properties of different black hole binaries.

Interstellar absorption plays an important role in spectrometric and photometric observations.
So does interstellar polarization in polarimetry.
The magnitude of interstellar polarization can significantly exceed intrinsic polarization, which makes elimination of ISM contribution a major problem.
Methods that rely on observing polarization of field stars proved their usefulness, yet the most efficient one is to observe X-ray binaries in their quiescence in addition to the field stars.
No comprehensive database of quiescent polarization exists, and a few known measurements show that sources in quiescence can have variable polarization.
If a quiescent source exhibits no intrinsic polarization, its observed polarization is the best estimate of the \gls{ISM} polarization in the source direction.
If otherwise, it provides an insight in the emission processes in the low-luminosity state and its possible connection to the hard state, or even permits an independent estimate of the orbital parameters.

Little is known about short-term intrinsic polarization variability of X-ray binaries.
The transient sources exhibit multiple types of periodic and quasi-periodic oscillations of different magnitude and origin in broad energy range, including optical and infra-red.
It is natural to expect some of these features to also affect intrinsic polarization.
However, polarimetry requires significantly more time to produce one measurement compared to simple photometry, which effectively limits the time resolution.
The magnitude of possible intrinsic polarization modulations requires long exposure times, which further complicates the process.
Improving the operation of existing polarimeters (such as \DUF) can increase their time resolution, yielding more information about the properties of the accretion process.


Finally, optical/near-infrared polarimetric data can augment photometric data. 
X-ray binaries show dramatic changes in their spectra during state transitions.
The changes in colour can be related to the changes in polarization degree and angle, presenting a full picture of the evolution of emitting components of X-ray binary throughout the outburst.
With quasi-simultaneous photometric/polarimetric data at hand, it may be possible to finally resolve the debate around the nature of the hard-state non-thermal component observed in the ONIR energy range. 
The upcoming launch of the Imaging X-ray Polarimeter Explorer in the end of 2021 will allow for simultaneous polarimetric observations in the ONIR and X-ray bands, opening a new window to exploring properties of accreting compact objects.
