\chapter{Summary of the original publications}
\let\oldsection\thesection
\renewcommand{\thesection}{\Roman{section}} 

\section{Double Image Polarimeter—Ultra Fast: Simultaneous Three-color ($BVR$) Polarimeter with Electron-multiplying Charge-coupled Devices }
We describe a new instrument capable of high-precision (up to 10$^{-6}$) polarimetric observations simultaneously in three passbands ($BVR$). 
This instrument is a result of collaboration between the University of Turku (Finland) and the Leibniz Institute for Solar Physics (Germany).
\DUF\ is built on the foundation of \DP\ polarimeters and makes use of much better hardware, including electron-multiplying charge-coupled device (EM CCD) cameras with high efficiency and fast image readout. 
We give technical descriptions of the control software, which was designed by the author of the thesis, discuss different operational regimes (polarimetric and photometric), and present first polarimetric results obtained with the help of the \gls{NOT}.

\section{High-precision optical polarimetry of the accreting black hole \VCYG\ during the 2015 June outburst}
In this paper we present high-precision polarimetric observations of a \gls{LMXB} \VCYG\ in $BVR$ filters during its outburst and in quiescence using \DP. 
We estimate the \gls{ISM} polarization using field stars and obtain intrinsic polarization of \VCYG, which is variable.
We apply statistical methods to demonstrate that the changes in polarizations are significant.
Its polarization \gls{SED} peaks in $V$-filter, reaching 1.1\%, and \gls{PA} in $R$-filter gradually changes by 30$^\circ$ over the course of the outburst.
We discuss the origin of polarized radiation and argue against the jet scenario.
We suggest that the likely source of polarization is either a combination of electron scattering and absorption in a flattened envelope or outflow surrounding the source, or scattering of disc radiation in mildly relativistic polar outflow.

\section{Evolving optical polarisation of the black hole X-ray binary \MAXI}
We present polarimetric observation of \gls{LMXB} \MAXI\ in $BVR$ filters during its 2018 outburst, obtained with the help of \DP.
We report a small and wavelength-dependent intrinsic polarization (0.3$-$0.7\%), which changes by $\sim 0.1\%$ during the course of observation campaign.
We suggest that the non-thermal component (jet or hot flow) observed in the hard state is unpolarized, and the polarization radiation may originate from the irradiated disc or from the scattering of disc radiation in the optically thin outflow.

\section{Disc and wind in black hole X-ray binary \MAXI\ observed through polarized light during its 2018 outburst }
We describe the first complete polarimetric data set of the entire outburst of an \gls{LMXB}.
Using the results of our previous work, we discuss the constraints for geometry and radiative mechanisms of \MAXI.
We report small intrinsic polarization ($\sim 0.15\%$ in $B$) in the soft state, which is likely produced by the irradiated disc.
We find a correlation between non-zero intrinsic polarization and presence of accretion winds, which suggests the origin of polarized radiation is scattering of the non-thermal (hot flow or jet base) radiation in an equatorial wind. 
We also note that the intrinsic \gls{PA} coincides with the jet position angle.

\section{Colors and patterns of black hole X-ray binary \GX}
In this paper we analyse a large data set of \gls{ONIR} light curves of \GX, which cover multiple regular and failed outbursts.
We use the soft state data to determine the extinction in the direction of the source and colour temperature of the disc.
With the help of \glspl{CMD} we demonstrate that various spectral states of regular outbursts occupy similar regions on the diagram, and that transitions between the states proceed along the same tracks despite substantial differences in the morphology of the observed light curves.
Using the soft state data, we subtract the contribution of the accretion disc during hard states and state transitions, obtaining \gls{ONIR} spectra of non-thermal component.
Using radio and \gls{midIR} data, we show that the radio to optical spectrum can be modeled using three components corresponding to the jet, hot flow, and irradiated accretion disk. 

\section{Superhump period of the black hole X-ray binary \GX }
We study timing properties of \gls{ONIR} light curves of \GX.
We apply two different time series analysis techniques to the soft state data and uncover prominent oscillations with an average period $P = 1.772 \pm 0.003$~d, which is offset from the measured orbital period of the system by 0.7\%
This is a signature of superhumps -- optical modulations caused by a 3:1 resonance in the disc, originally observed in cataclysmic variables.
We compare \GX\ to other \glspl{BHXRB} that are known to exhibit superhumps and  discuss the implications of this finding in the context of superhump theory.

\clearpage
\renewcommand{\thesection}{} 

\section{The author's contribution to the publications}

\subsection*{Paper I. Double Image Polarimeter—Ultra Fast: Simultaneous Three-color ($BVR$) Polarimeter with Electron-multiplying Charge-coupled Devices}
The author contributed to the construction of the polarimeter, designed and implemented the control software for the instrument, configured remote observation mode and helped to deploy the polarimeter to the \gls{NOT}.
The author participated in the data acquisition and analysis, wrote sections of the manuscript about control hardware and software and made contributions to other parts of the paper. 

\subsection*{Paper II. High-precision optical polarimetry of the accreting black hole \VCYG\ during the 2015 June outburst}
The author carried out all data analysis and statistical tests, produced figures and tables, and wrote the majority of the manuscript.

\subsection*{Paper III. Evolving optical polarisation of the black hole X-ray binary \MAXI}
The author contributed to the computation of the \gls{ISM} polarization and produced the figure depicting \gls{PD} and \gls{PA} of the source and field stars.
The author described the statistical methods and carried out statistical tests.
The author also made contributions to other sections of the manuscript.

\subsection*{Paper IV. Disc and wind in black hole X-ray binary \MAXI\ observed through polarized light during its 2018 outburst }
The author processed data, produced figures and tables, and carried out statistical tests. 
The author also wrote the majority of the manuscript.

\subsection*{Paper V. Colors and patterns of black hole X-ray binary \GX}
The author carried out data analysis and model fitting, produced figures and tables.
The author documented the data processing routine, described obtained results and wrote most of the manuscript.

\subsection*{Paper VI. Superhump period of the black hole X-ray binary \GX }
The author discovered the superhump period in the publicly available data, carried out data analysis, performed model fitting and evaluated power spectral densities to support his finding.
The author produced figures and tables and wrote the majority of the manuscript.



\renewcommand{\thesection}{\oldsection} 
