\chapter{Optical polarimetry of accreting black holes}
Polarimetry is an emerging and powerful tool that can be used to identify emitting components in different outburst phases.
Polarization spectrum is a product of energy spectrum and polarization profile, each of which carries information about emission and scattering mechanisms responsible for the produced light.
Whenever accreting system undergoes a dramatic change in the geometry of its emitting components, such as state transition during outburst, polarization spectrum is expected to reflect this change.

One of the major sources of polarization is synchrotron emission.
Jets can produce highly linearly polarized (up to 70\%) \gls{ONIR} emission with a soft spectrum \citep{Zdziarski2014} if the magnetic field is ordered.
Another source of synchrotron radiation, hot inner flow, likely contributes little to no \gls{ONIR} polarization owing to the structure of its magnetic field and potential depolarization caused by Faraday rotation (\citealt{Poutanen2014a}; \paperIII).
Accretion disc can produce moderate polarization with wavelength-dependent polarization angle.
The polarization depends on the optical depth \citep{Cha60, Sobolev1963, Beloborodov1999} and is either parallel or perpendicular to the disc axis.
Significant absorption opacity can rotate the polarization angle by 90$^\circ$ in limiting cases \citep{Nagirner1962, Gnedin1978}, or even result in a smoothly varying with wavelength angle, similar to what is observed in Be stars (\citealt{Poeckert1979}; hints of this type of behaviour were observed in the soft state of \MAXI, see \paperIII).

Scattering of radiation produced by any of the emitting components (such as accretion disc, hot accretion flow, or relativistic jet) may introduce linear polarization even if source emission is intrinsically unpolarized.
Scattering of the disc emission in slow wind yields small polarization, perpendicular to the disc axis \citep{Gnedin1997}.
Non-thermal emission from the hot flow or base of the jet can be scattered by the wind as well, producing up to 30\% linear polarization depending on the system inclination and the scattering fraction (\citealt{Sunyaev1985}; \paperIV).
If the scattering occurs in a relativistic outflow, the resulting polarization can reach 20\% parallel to the symmetry axis \citep{Beloborodov1998}.


Even though any of the emitting components can independently produce highly-polarized radiation, the overall intrinsic polarization is usually heavily diluted by non-polarized components and is therefore small, of the order of few per cent.
Observed polarization of a source is a combination of intrinsic and \gls{ISM} polarization (see Sect.~\ref{sec:pol_ism}), the latter usually increases proportionally to the interstellar extinction.
Intrinsic polarization can be revealed by subtracting the best estimate of \gls{ISM} polarization, which can be obtained by, e.g., observing a large sample of field stars \paperssp{II}{III}, or \gls{BHXRB} in its quiescent state (but note that some systems show variable quiescent polarization, \citealt{Dolan1989, Dubus2008, Russell2016}, \inprepmaxi).

\VCYG\ and \MAXI\ have been extensively monitored in polarized light during their outbursts. 
Both systems showed small, but statistically significant variable intrinsic polarization.
However, their behaviour in quiescence differ significantly.
\VCYG\ shows \gls{ISM}-level linear polarization one \paperIIp\ and four years \paperIp\ after its 2015 outburst.
\MAXI, on the contrary, exhibits substantially larger intrinsic polarization, misaligned with \gls{ISM} polarization in its direction (\paperI; \inprepmaxi).

Several other \glspl{BHXRB} were observed in quiescence as well.
\SSiv\ and \VSGR\ both show no detectable intrinsic polarization \paperIp.
\iA\ exhibits orbital modulations of its linear polarization (up to 2\% in 1988, see \citealt{Dolan1989}, but only $\sim0.2$\% when observed 15 years later, \citealt{Dubus2008}), while \SwiftJxiii\ shows high linear polarization, which significantly exceeds the maximum \gls{ISM} polarization produced by the dust in the direction of the source \citep{Russell2016}.

It is yet unclear why some systems exhibit substantial intrinsic polarization in quiescence, while others do not.
\citet{Russell2016} argue that quiescent intrinsic polarization is a sign of jet activity, especially if accompanied by photometric variability \citep{Russell2006, Gallo2007, Plotkin2016}.
The peculiar case of \MAXI\ can be an argument in favour of disc emission scattering as the mechanism for quiescent polarization (\inprepmaxi).


It is worth noting that \gls{ONIR} polarimetry of accreting \glspl{BH} is a difficult task. 
In the outburst \glspl{BHXRB} are relatively bright but exhibit very small intrinsic polarization (if any), while in quiescent they are very faint and show no intrinsic polarization (with a few notable exceptions discussed above).
In both cases large telescopes and/or long integration times are required for achieving accuracy levels that allow for reliable measurement of intrinsic polarization.
The onset of an outburst is largely unpredictable, which further complicates observation process, as there are only a few polarimetric instruments capable of monitoring \gls{ONIR} targets of opportunity as soon as they brighten.

Even though there is evidence suggesting intrinsic polarization changes with spectral state, it is still unclear if intrinsic polarization varies with orbital phase.
With typical orbital periods of \glspl{LMXB} of several hours to several days, multiple polarimetric observations with sufficient accuracy per night are required to test orbital variability.
This has not been achieved so far for sources other than \iA.

In conclusion, accreting \glspl{BH} require a systematic polarimetric study, covering both outbursts and quiescence.
A comprehensive overview of polarimetric properties of \glspl{BHXRB} may help to not only better understand the origin of non-thermal \gls{ONIR} excess observed in the hard state, but also identify emission mechanism active in quiescence, at the same time providing valuable input on the geometrical properties and orientation of emitting/scattering regions.




