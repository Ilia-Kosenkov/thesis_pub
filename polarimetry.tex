\chapter{Measuring polarization of astrophysical sources}
Polarization is a fundamental property of light, yet it is quite hard to measure, because most of detectors are not sensitive to polarization.
Stokes parameters $Q$, $U$, and $V$ cannot be easily measured directly, however, there are numerous methods of measuring $I$ -- intensity.
Polarimetric techniques usually involve i) modulation of the polarized fraction of radiation and ii) measuring the effect of this modulation on the total intensity registered by a detector.
The relationship between incident radiation $\vctr{S}_\mrm{inc}$ and radiation detected after propagating through an instrument $\vctr{S}_\mrm{det}$ can be conveniently expressed in terms of Mueller calculus:
\begin{equation}
    \vctr{S}_\mrm{inc} = \mtrx{M}_\mrm{inst}^{-1} \vctr{S}_\mrm{det},
\end{equation}
where $\mtrx{M}_\mrm{inst}$ describes the optical properties of the (ideal) instrument.
Each element of the optical system can be represented by its own Mueller matrix, $\mtrx{M}_i$, such as
\begin{equation}
    \mtrx{M}_\mrm{inst} = \prod\limits_i \mtrx{M}_i.
\end{equation}
For the \gls{ONIR} polarimetry, a typical instrument consists of a telescope and \textit{polarimeter} -- a dedicated device which is capable of detecting polarization of the received radiation (usually, either linear, or circular polarization, or both).
Some elements of the instrument's optical system introduce undesired, potentially variable polarization, which contaminates the incident light, contributing to the \textit{instrumental} polarization.


\section{Telescope}
Telescopes are usually the main source of instrumental polarization, which cannot be eliminated completely.
Unfortunately, some telescope designs lead to substantial, but also variable instrumental polarization.
The magnitude of this contaminating polarization depends on, for example, telescope orientation.
Such instruments significantly complicate polarimetric observations, especially of faint or low-polarization targets.

What is causing disruptive instrumental polarization of the telescopes? 
As discussed above, one of the main sources of polarization is asymmetry.
Thus, axisymmetric foci (such as Cassegrain and Gregorian) should exhibit much smaller instrumental polarization compared to non-axisymmetric ones (e.g, Nasmyth or Coud\'{e}, \citealt{Serkowski1974, AstronomicalPolarimetry}).
Even in a perfectly symmetric optical system instrumental polarization of on-axis image can be caused by the axis-asymmetric imperfections of the mirror surfaces of the telescope, which can never be polished ideally. 
The magnitude of these effects is relatively small, however in the worst cases instrumental polarization can reach up to 1\%.
The methods of instrument polarization determination are discussed in Sect.~\ref{sec:data_red}.


\section{Polarimeter}
Polarimeters are instruments of special design that transform and measure incident (un-)polarized radiation.
Polarimeters typically consist of three principal components: a modulator, an analyzer, and a detector \citep{Serkowski1974,AstronomicalPolarimetry,Berdyugin2019}.

\subsection{Modulators}
\label{sec:pol:mods}
Modulation of polarization of the incident radiation is required for accurate optical/near-infrared polarimetry \citep{Serkowski1974,AstronomicalPolarimetry}, and is commonly achieved by introducing a retarder, which creates an additional phase shift between orthogonally polarized components of the radiation.
Two types of retarders with constant phase shift are widely used: \gls{HWP}, which introduces $\delta = \pi$ phase shift and rotates linear polarization, and \gls{QWP}, which produces $\delta = \pi/2$, transforming circular polarization into linear and vice versa.

Modulators can be separated into three main categories depending on their optical properties: with fixed phase shift, with variable phase shift, and with constant delay but variable optical axis position.
The fixed phase shift modulators are wave plates made of birefringent crystals, polymers.
Photoelastic or piezoelectric modulators made of non-birefringent materials, which alter their phase shift magnitude in response to variable external stress, Pockels/Kerr cell-based electro-optic and nematic liquid crystals, susceptible to changes in the applied voltage, are examples of variable phase shift modulators.
Ferro-electric liquid crystals exhibit variable optical axis position \citep{Berdyugin2019}.


\subsection{Analyzers}
Analyzers are essential for separation of orthogonally linearly polarized light components.
The simplest analyzers -- polaroids -- are absorptive and exhibit strong absorption of radiation, polarized parallel to the optical axis (polaroid and Nicole prism; \citealt{AstronomicalPolarimetry}).
Alternatively, one of the linearly polarized light components can be reflected (Glan--Thompson prism).
They are used in the single-beam polarimeters and allow detectors to measure magnitude of linear polarization of only one orthogonal component at a time.
The disadvantage of single-beam analyzer is the loss of the half of the incident light intensity.

Two-beam analyzers utilizes the birefringence property of crystals.
Such crystals have different refractive indices for light polarized perpendicular to the principal direction (ordinary ray) and parallel to that direction (extraordinary ray).
As a result, the incident radiation is split into two orthogonally polarized rays travelling along slightly different optical paths.
This property is leveraged in the double-beam analyzers, including plane-parallel calcite plate, Savart plate and Wollaston prism \citep[e.g.,][]{Serkowski1974, AstronomicalPolarimetry,Berdyugin2019}.
Double-beam analyzers allow simultaneous registration of both orthogonally polarized light beams which is necessary for efficient and high-precision polarimetry.

\subsection{Detectors}
\label{sec:pol:detectors}
Detector is the final component of any instrument.
\gls{ONIR} polarimeters require highly sensitive detectors, such as \gls{EM} \glspl{CCD}, \gls{CMOS}, their combination in the form of \gls{sCMOS}, photo-multipliers and avalanche photo-diodes are all used for \gls{ONIR} polarimetry.

Each detector type has its own advantages and disadvantages. 
Photo-multipliers and avalanche photo-diodes represent a family of single-cell detectors.
They are capable of registering fairly large fluxes, retaining nearly-linear response up to 10$^8$ $e\,$s$^{-1}$ \citep{Berdyugin2019}.
Photo-multipliers show quite low wavelength-dependent \gls{QE} $\sim 10-40\%$ compared to \gls{QE} of avalanche photo-diodes ($\sim 80\%$ in the infrared, however, much lower in optical).
Both detector technologies allow for extremely fast readout speeds, though avalanche photo-diodes exhibit substantial dark currents, owing to the extra bias voltage, applied to it.
Thus, detectors of these families are best suited for high-precision/fast polarimetry of bright targets, if paired with a high-frequency (such as photoelastic) modulator.

The \gls{CCD} and \gls{CMOS} detectors represent a family of multi-cell devices.
They usually consist of grids of many thousands of sensitive elements (pixels), which allows capturing of detailed images of the sky in polarized light at the cost of readout speed.
The technology behind pixel design significantly limits the maximum flux that can be recorded by a single pixel without contaminating the whole image. 
If this limit is exceeded, pixel `saturates' excessive charge can `leak' into adjacent pixels, or otherwise be carried over unsaturated pixels, destroying parts of the readout image.
With higher quantum efficiency and ability to observe the target, surrounding sky, and, possibly, field stars simultaneously on the same detector, \gls{CCD} and \gls{CMOS} are best suited for studying faint objects.
Low readout speeds are well-paired with slower polarization modulators, such as discretely rotating half-/quarter-wave plates.

\gls{CCD} cameras are well suitable for polarimetry with dual-beam analyzers.
A \gls{CCD} can simultaneously record both orthogonally polarized images of the star after an analyzer splits incoming radiation into two light rays.
The plane-parallel calcite also transforms the radiation coming from the sky, superimposing orthogonally polarized images of the sky onto each image of the target. 
This effectively optically eliminates polarization of the sky and allows polarimetric measurements using simple aperture photometry techniques \citep{Berdyugin2019}.
The dual-beam \gls{EM} \gls{CCD} polarimeters are widely used and proved to be efficient for various tasks, including monitoring of compact transient objects.
The notable examples of such polarimeters are RINGO3 \citep{Arnold2012} and its successor \gls{MOPTOP}, \gls{GASP}, \gls{GPP}, DIPol-family polarimeters \DP\ and \DUF.
There are also multi-mode instruments, among which are \gls{ALFOSC}, \gls{EFOSC}, \gls{FORS}, that support dual-beam \gls{CCD} polarimetry as one of their regimes.











