\chapter{A brief history of black holes}

The first idea of a celestial object so massive that light cannot escape its surface was expressed in 1783 by John Michell, a clergyman who received geological education in England \citep{Pounds2014}.
In his letter published in 1784 \citep{Michell1784}, he estimated that a free-falling from infinity body can reach the speed of light\footnote{Michell assumed the speed of light to be $\sim 10^4$ times the Earth's orbital velocity. 
} at the surface of an attracting celestial object if this attractor is 500 times larger than the Sun, but has Sun's density.
He wrote:
\begin{quotation}
    \textit{
        16. Hence, according to article 10, if the semi-diameter of a sphere of the same density with the Sun were to exceed that of the Sun in the proportion 500 to 1, a body falling from an infinite height towards it, would have acquired at its surface a greater velocity than that of light, and consequently, supposing light to be attracted by the same force in proportion to its vis inertia, with other bodies, all light emitted from such a body would be made to return towards it, by its own proper gravity.
    }
\end{quotation}
In simple words, Michell suggested that radiation, emitted by such star-like bodies, is unable to escape their vicinities, rendering them completely invisible for an observer.
His theory, while revolutionary, was based on the assumption that radiation is represented by 'corpuscles', and that gravity of a massive celestial body can affect and slow down emitted light corpuscles.

17 years before his letter was published, Michell wrote another fascinating theoretical manuscript.
In his very first astronomical publication \citep{Michell1767}, he argued (applying statistical methods) that stars occur in groups more often than a random distribution predicts.
According to Michell, stars are drawn together owing to the gravitational pull, and some of the visually close stars may be part of one system.
The possible existence of binary systems allowed him to predict in his later work \citep{Michell1784}, that if a star, invisible to an observer due to its extraordinary mass, is part of a binary system, and its companion is a normal star, then the presence of the hidden star can be inferred from the irregularities in the motion of the visible star:
\begin{quotation}
    \textit{
        29. If there should really exist in nature any bodies, whose density is not less than that of the Sun, and whose diameters are more than 500 times the diameter of the Sun, since their light could not arrive at us; or if there should exist any other bodies of a somewhat smaller size, which are not naturally luminous; of the existence of bodies under either of these circumstances, we could have no information from light; yet, if any other luminous bodies should happen to revolve about them we might still perhaps from the motions of these revolving bodies infer the existence of the central ones with some degree of probability...
    }
\end{quotation}


Twelve years after Michell's publication, another prominent scientist -- Pierre-Simon de Laplace of France -- suggested in his manuscript that an object 250 times the size of the Sun, but with the density equal to that of the Earth, would produce so much gravitational attraction that the light could not escape from its surface.
He concluded that the largest and most massive objects in the Universe may be completely invisible.
Laplace later presented a mathematical proof of his theory in the form of an essay \citep{Laplace1799}.


It is unclear whether Laplace was influenced by Michell's work \citep{Montgomery2009}.
Michell's idea of invisible objects was merely a byproduct of his research, while Laplace focused on these hypothetical objects and even attempted to prove their existence.
Unfortunately, both scientists considered only enormously large star-like objects with densities close to that of the Sun or the Earth.
It took the scientific community more than 100 years to uncover the nature of light and understand how extreme gravity can affect and bend light.

Albert Einstein developed his theory of General Relativity in the beginning of XX century \citep{Einstein1915}.
A year later, a peculiar solution of Einstein's field equations for a point mass was found by Karl Schwarzchild -- his solution had two special points (mathematical singularities).
One of the singularities is located at $R_\mrm{Schw} = 2GM/c^2$, at which some of the equations' terms become infinite (here $G$ is the gravitational constant, $M$ is the mass of an object, $c$ is the speed of light).


The work continued for the next 50 years. 
During this period, important advancements were made in the field of stellar evolution, including fundamental work by Subrahmanyan Chandrasekhar (\citealt{Chandrasekhar1931a}, but see also \citealt{Stoner1930} and \citealt{Landau1932}), where he showed there is no stable solution for white dwarfs over certain limiting mass.
Julius Oppenheimer and George Volkoff later arrived to a similar conclusions for the neutron stars \citep{Oppenheimer1939}, utilizing solutions obtained by Richard Tolman \citep{Tolman1939}.
Oppenheimer and Snyder later described a non-stationary solution of continuous, ever slowing down collapse of a massive star \citep{Oppenheimer1939a}, which eventually contracts to the size of its Schwarzchild radius, thus introducing a possible formation mechanism for objects that prevent all radiation from escaping their vicinity, interacting only through gravitational pull exerted on other bodies.


General solutions for gravitational field of a point mass were obtained by Roy Kerr \citep{Kerr1963} and Ezra Newman \citep{Newman1965}, which included angular momentum and electric charge of the body.
Around the same time, the term -- \textit{black hole} -- was coined by John Wheeler during his lecture in 1967 \citep{Pounds2014}, after a student allegedly suggested it a few weeks before.
According to a number of sources, this term has been already in use (during meetings and symposia) as early as in 1963, but no clear record of its origin can be found. 
The properties of the \textit{event horizon} \citep{Finkelstein1958} were revealed in a series of papers \citep[e.g., ][]{Israel1968, Carter1971}, which resulted in a formulation of the `no-hair theorem': a stationary black hole is completely described by its mass, angular momentum, and electric charge.

The discovery of the first bright extrasolar X-ray sources, Sco~X\nobreakdash-1 \citep{Giacconi1962}, Cyg~X-1 \citep{Bowyer1965}, and Cen~X\nobreakdash-3 \citep{Chodil1967}, was a pivotal point in the history of astrophysics.
Cyg~X-1, which was extensively monitored in X-rays \citep{Overbeck1968, Schreier1971}, demonstrated hard spectrum and substantial variability on the time-scale of seconds, which suggested the emission region was relatively small \citep{Pounds2014}.
The identification of the optical counterpart provided the distance to the spectroscopic binary system, and the mass of the X-ray source was estimated to be $\ge 10 M_\odot$ (\citealt{Paczynski1974}; latest measurements constraint the mass to be $\sim 21 M_\odot$, \citealt{Miller-Jones2021}), well above the theoretical limit of neutron star, making Cyg~X\nobreakdash-1 the first stellar mass black hole candidate.

Today, many dozens of Galactic X-ray sources are known, and every year brings new discoveries.
Black holes (and black hole candidates) of stellar masses are found in the binary systems, revealing themselves to the external observer through the process of accretion -- an effective mechanism of transforming matter into energy \citep{AccretionPower}.
For some time, black holes were studied through observing emission patterns (shape of the spectrum, variability and timing properties), produced by the matter in the vicinity of black holes.
In 2015, the first detection of gravitational waves from the binary black hole merger (GW150914) was made by the \gls{LIGO}, marking the beginning of a new era of gravitational astronomy. 
Gravitational waves, predicted by Einstein in 1916 \citep{Einstein1916}, provide a separate tool for determination of masses of compact objects, and allows tracing the final moments of the binary system evolution.
Since its upgrade, \gls{LIGO} together with the Virgo interferometer consistently report detections of black hole -- black hole mergers, as well as neutron star -- neutron star collision events \citep[GW170817,][]{Abbott2017}.
While there has been no detection of black hole -- neutron star merger to date, gravitational wave from black hole -- black hole mergers with `low-mass' components (e.g., 7 and 12 $M_\odot$ in GW170608, \citealt{Abbott2017b}), or with large mass ratio (GW190412, \citealt{Abbott2020a}) have been observed, suggesting black hole -- neutron star event can still be measured by \gls{LIGO}/Virgo when gravitational waves from such collision finally reach Earth.

Despite the focus of the present work being the stellar mass black holes, it is important to mention the first direct image of the supermassive black hole and its vicinity \citep{EHTC2019}.
The resolved image of an accreting black hole and its `shadow' in the centre of M87 galaxy provides yet another confirmation of the existence of black holes and sheds more light on the properties of their immediate surroundings in the radio wavelengths.


