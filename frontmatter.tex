
%
% set document language based on our selection in the preamble
% 
\iffin
 \selectlanguage{finnish}
\else
 \selectlanguage{english}
\fi

%
% Set the following info for title back page, abstracts, etc.
%
\faculty{Faculty of Science}{Matemaattis-luonnontieteellinen tiedekunta}
\dept{Department of Physics and Astronomy}{Fysiikan ja t\"ahtitieteen laitos}
%\subject{High-energy astrophysics}{Suurenergia-astrofysiikka}
\subject{Astronomy}{Tähtitiede}
\programme{Doctoral programme in Physical and Chemical Sciences}{Fysikaalisten ja kemiallisten tieteiden tohtoriohjelma}
\mnth{June}{Kesäkuu}
% if you have an image on the book cover uncomment and fill the info
\coverinfo{ESO/L. Cal\c{c}ada/M.Kornmesser}\imgcovertrue



%%%
%%% Title page
%%%
\TitlePage{Publication Series information should go here}

%%%
%%% back of the title page
%%%
% parameters: #1 1st supervisor, #2 2nd, #3 1st reviewer, #4 2nd, #5 opponent
\TitleBack{Prof. Juri Poutanen\\Department of Physics and Astronomy, University of Turku, \\Turku, Finland}
{Dr. Alexandra Veledina\\Department of Physics and Astronomy, University of Turku, \\Turku, Finland}
{Doc. Andrei Berdyugin\\Department of Physics and Astronomy, University of Turku, \\Turku, Finland}
{Assoc. Prof. Emrah Kalemci\\Faculty of Engineering and Natural \\ Sciences, Sabanci University, \\Orhanli-Tulza, Istanbul, Turkey}
{Dr. Stefano Bagnulo\\Armagh Observatory \& Planetarium, \\College Hill, Armagh, Northern Ireland, United Kingdom}
{Dr. Eugene Churazov\\Max Planck Institute for Astrophysics, \\Karl-Schwarzchild-Str. 1, \\D-85741 Garching, Germany}


%%%
%%% Dedication can simply be handled as follows
%%%
% TODO: update with acknowledgements
\thispagestyle{empty}
\vspace*{\fill}
\begin{flushright}

\textit{To my wife Anastasiia}
\end{flushright}
\newpage


%%%
%%% Abstracts both in english and finnish
%%%
% the order should be such that the 1st one corresponds to the language
% of the dissertation. Remember to turn on/off the "second language"
% before and after the 2nd abstract.
\begin{abstract}
    Black holes are among the most unusual objects in the Universe.
    Powered through accretion, they are strong sources of radiation in a broad energy range, from radio to hard X-rays.
    Stellar-mass black holes manifest themselves in binary systems, when their companion star -- main sequence or giant -- starts to lose matter, which is then captured and accreted by the black hole.

    Accreting black hole binaries are complex systems.
    We observe emission from multiple components, such as accretion disc, hot accretion flow (corona), jet, and even donor star.
    A large number of accreting black holes are transient sources -- they undergo periods of violent activity, increasing luminosity by several orders of magnitude.
    Throughout the course of an active phase -- an outburst -- the spectral energy distribution and relative contribution of each component can change dramatically, resulting in a gradual evolution of the observed spectra of the black hole binary transient.
    
    In the first part of the thesis I describe the nature of the black hole binaries, focusing on the emission mechanisms. 
    Using the archival photometric data of \GX\ I demonstrate how observed optical and infrared spectral properties of the source can be explained with a three-component jet, hot flow and accretion disc model.
           
    In the second part I discuss polarization mechanisms.
    Polarization is a fundamental property of light and it carries information about the geometrical structure of the source and scattering or polarizing media.
    I review processes that can produce polarized (or polarize unpolarized) radiation in the accreting black hole binaries and interstellar medium.
    
    In the third part I introduce the novel optical polarimeter (DIPol-UF), which was built in Tuorla Observatory as part of an international collaboration.
    I outline the challenges of remotely operating high-precision polarimeter and describe the control software that I developed specifically for this instrument.

    Finally, I discuss the properties of intrinsic polarization of low-mass X-ray binaries based on polarimetric data of \VCYG\ and \MAXI.
    Both objects showed small and variable intrinsic polarization during an outburst, with polarization angle coinciding with jet position angle.
    I demonstrate how high-precision polarimetry can augment photometric and timing studies of X-ray binaries, shedding more light onto the nature of optical emission in these objects.
    

\mbox{}\\\noindent
KEYWORDS: polarization, accretion, black hole binaries
\end{abstract}

\fintrue % turn on finnish temporarily
\begin{abstract}
    Mustat aukot ovat eräitä maailmankaikkeuden epätavallisimmista kohteista. 
    Aineen kertymisen vuoksi ne ovat voimakkaita säteilyn lähteitä laajalla energiavälillä radioaalloista röntgensäteisiin. 
    Tähtienmassaiset mustat aukot ilmenevät kaksoistähti-järjestelmissä, kun niiden pääsarja- tai jättiläisvaiheessa oleva kumppanitähti alkaa luovuttamaan ainetta, joka kertyy mustaan aukkoon.

    Kerryttävän mustan aukon sisältävät kaksoistähdet ovat monimutkaisia järjes-telmiä.
    Havaitsemme säteilyä useista lähteistä, kuten kertymäkiekosta, kuumasta kertymävirtauksesta, suihkusta ja jopa kumppanitähdestä. 
    Useat kerryttävät mustat aukot ovat ajoittain havaittavia kohteita: Ne käyvät läpi väkivaltaisia jaksoja, jolloin niiden kirkkaus kasvaa usealla kertaluokalla.
    Koko aktiivisen ajan, eli purkauksen ajan, energian spektrijakauma ja sen jokaisen komponentin suhteellinen osuus voi vaihdella hurjasti, mikä johtaa musta aukko -kaksoisjärjestelmän havaitun spektrin asteittaiseen muuttumiseen.

    Väitöskirjan ensimmäisessä osassa kuvaan mustan aukon sisältävien kaksoistähti-järjestelmien luonnetta keskittyen säteilymekanismeihin. 
    Käyttäen arkistoituja fotometrisiä havaintoja kohteesta \GX\ osoitan kuinka havaittu optinen ja infrapunaspektri voidaan selittää kolmen komponentin mallilla, joka koostuu suihkusta, kuumasta virtauksesta ja kertymäkiekosta.

    Toisessa osassa käsittelen polarisaatiomekanismeja. 
    Polarisaatio on valon perustavanlaatuinen ominaisuus, ja se sisältää tietoa kohteen geometrisestä rakenteesta sekä sirottavasta ja polarisoivasta väliaineesta. 
    Käyn läpi prosesseja, jotka voivat tuottaa polarisoitunutta säteilyä (tai polarisoida polarisoitumatonta sellaista) kerryttä-vissä musta aukko -kaksoisjärjestelmissä tai tähtienvälisessä aineessa.

    Kolmannessa osassa esittelen uuden optisen polarimetrin (DIPol-UF), joka rakennettiin Tuorlan Observatoriossa osana kansainvälistä yhteistyötä. 
    Hahmottelen erittäin tarkan polarimetrin kaukokäyttöön liittyviä haasteita ja kuvailen tälle laitteelle kehittämääni ohjausohjelmistoa.

    Lopuksi käyn läpi pienimassaisten röntgenkaksoistähtien luontaisen polarisaation ominaisuuksia perustuen polarimetrisiin havaintoihin kohteista \VCYG\ ja \MAXI. 
    Molemmat kohteet osoittavat pientä ja vaihtelevaa luontaista polarisaatiota purkauksen aikana siten, että polarisaatiokulma sopii yhteen suihkun sijaintikulman kanssa. 
    Osoitan kuinka erittäin tarkka polarimetria voi olla lisänä röntgenkaksoistähtien fotometrialle ja ajoitustutkimuksille valaisten lisää näiden kohteiden optisen säteilyn luonnetta. 

\mbox{}\\\noindent
ASIASANAT: polarisaatio, kertyminen, musta aukko, kaksoistähti
\end{abstract}
\finfalse % turn off finnish

%%%
%%% Acknowledgements is also without a number
%%%
% TODO: Acknowledgements

\chapter*{Acknowledgements}
\addcontentsline{toc}{chapter}{Acknowledgements}
\thispagestyle{plain}
I started this journey almost six years ago, when I decided that life without scientific exploration and challenge is not for me.
The path was long and difficult, but the outcome is exciting and satisfying.
Throughout my journey I found myself surrounded by people who supported me.

First of all, I owe my deepest gratitude to my supervisors, Prof. Juri Poutanen, Dr. Alexandra Veledina, Doc. Andrei Berdyugin.
They shaped me as a researcher, guided me through the early stages of my career, and provided support that is invaluable for a student.
This thesis would not be possible without them.

I am grateful to Assoc. Prof. Emrah Kalemci and Prof. Stefano Bagnulo for rigorous pre-examination of my thesis.
Their feedback and comments helped me to improve my work.
I thank Dr. Eugene Churazov for agreeing to become my esteemed opponent.

I would like to extend my gratitude to my collaborators, who influenced and contributed to the research presented in this thesis.
I am grateful to all of my co-authors, including Sergey Tsygankov, Vilppu Piirola, Valery Suleimanov and Svetlana Berdyugina.
This work would not be possible without the support provided by the University of Turku and Leibniz Institute for Solar Physics.
I am grateful to NORDITA KTH and NOT for research opportunities.

I would like to also thank my partners-in-crime, PhD students with whom I shared my path: Joonas, Anna, Tuomo, Vadim, Vlad, Juhani, Armin, Yasir.
Some of them inspired and taught me, some of them, I hope, drew inspiration and learned something from me.
I am also grateful to Pavel and Roberto.
All of them helped me grow both as a student and as a researcher.
I am tankful to the members of Tuorla Observatory (and FINCA), who welcomed and accepted me.

My life would have been much more difficult without my friends.
With some of them we share almost 20 years of history.
I am grateful to Sergey, Artyom, Semyon, Pavel and Natasha.

Even though we were separated by a great distance, I never felt lonely during these years.
My family, my mother Tamara and stepfather Gennadii, my grandmother Zinaida provided much needed support during my highs and lows. 
\fontencoding{T2A}\selectfont
Спасибо вам за всё!
\fontencoding{T1}\selectfont

I am deeply grateful to my wife Anastasiia for her love and understanding.
I walked this path knowing she is there for me and that I cannot let her down.

\mbox{}\\[2\baselineskip]
\begin{flushright}
17.08.2021 \\
\textit{\Author}
\end{flushright}
%
% optional author introduction box 
%
% TODO : Should we use it?
% \vfill %this will put the box at the bottom of the page
% \begin{infobox}{images/authorimage.png}
% \end{infobox}
\mbox{}\newpage


%%%
%%% ToC, automatically renamed based on selected language
%%%
\iffin
\def\contentsname{Sisällys}
\else
\def\contentsname{Table of Contents}
\fi
\tableofcontents

%%%
%%% Abbreviations is without a number
%%%
% \iffin
% \chapter*{Lyhenteet}
% \addcontentsline{toc}{chapter}{Lyhenteet}
% \else
% \chapter*{Abbreviations}
% \addcontentsline{toc}{chapter}{Abbreviations}
% \fi
% 
% your abbrevs should go here, as an example we'll use the entry environment
%
% \begin{entry}{ARRLA} %parameter is the widest label
% \item[AA] An Abbreviation
% \item[ARRLA] A Really Really Long Abbreviation
% \item[SS] Something something
% \end{entry}
{
\glssetwidest{WATCHDOGM}
% \footnotesize
\iffin
\printnoidxglossary[
    style=alttree,
    title=Lyhenteet,
    toctitle=Lyhenteet, 
    sort=word, 
    type=\acronymtype
]
\else
\printnoidxglossary[
    style=alttree,
    title=Abbreviations,
    toctitle=Abbreviations, 
    sort=word, 
    type=\acronymtype
]
\fi
}
%%%
%%% List of Publications is also without a number
%%%
\iffin
\chapter*{Artikkelit}
\addcontentsline{toc}{chapter}{Artikkelit}
\else
\chapter*{List of Original Publications}
\addcontentsline{toc}{chapter}{List of Original Publications}
\fi
This dissertation is based on the following original publications, which
are referred to in the text by their Roman numerals:

% ADS custom format is %g\\\\ %T.\\\\ %J, %Y; %V: %p.\n
% The template is:
% \item[I] Author(s) of the publication. Full title of the publication. Journal, publishing ear; issue number: pages.
\begin{entry}{III} % the parameter is the widest label
    \item[I] Piirola V., \textit{Kosenkov I. A.}, Berdyugin A. V., Berdyugina S. V., Poutanen J. \\ Double Image Polarimeter—Ultra Fast: Simultaneous Three-color ($BVR$) Polarimeter with Electron-multiplying Charge-coupled Devices.\\ 2021, The Astronomical Journal, 161, 20
    \item[II] \textit{Kosenkov I. A.}, Berdyugin A. V., Piirola V., Tsygankov S. S., Pall\'e E., Miles-P\'aez P. A., Poutanen J.\\ High-precision optical polarimetry of the accreting black hole \VCYG\ during the 2015 June outburst.\\ 2017, Monthly Notices of the Royal Astronomical Society, 468, 4362
    \item[III] Veledina A., Berdyugin A. V., \textit{Kosenkov I. A.}, Kajava J. J. E.,\\ Tsygankov S. S., Piirola V., Berdyugina S. V., Sakanoi T., Kagitani M., Kravtsov V., Poutanen J.\\ Evolving optical polarisation of the black hole X-ray binary \MAXI.\\ 2019, Astronomy and Astrophysics, 623, A75
    \item[IV] \textit{Kosenkov I. A.}, Veledina A., Berdyugin A. V., Kravtsov V., Piirola V., Berdyugina S. V., Sakanoi T., Kagitani M., Poutanen J.\\ Disc and wind in black hole X-ray binary \MAXI\ observed through polarized light during its 2018 outburst.\\ 2020, Monthly Notices of the Royal Astronomical Society, 496, L96
    \item[V] \textit{Kosenkov I. A.}, Veledina A., Suleimanov V. F., Poutanen J.\\ Colors and patterns of black hole X-ray binary \GX.\\ 2020, Astronomy and Astrophysics, 638, A127
    \item[VI] \textit{Kosenkov I. A.}, Veledina A.\\ Superhump period of the black hole X-ray binary \GX.\\ 2018, Monthly Notices of the Royal Astronomical Society, 478, 4710
\end{entry}
\clearpage

\iffin
\section*{List of publications not included in the thesis}
\else
\section*{List of publications not included in the thesis}
\fi
\begin{entry}{III}
    \item[] Kravtsov V., Berdyugin A. V., Piirola V., \textit{Kosenkov I. A.}, Tsygankov S. S., Chernyakova M., Malyshev D., Sakanoi T., Kagitani M., Berdyugina S. V., Poutanen J.\\ Orbital variability of the optical linear polarization of the $\gamma$-ray binary \LSI\ and new constraints on the orbital parameters.\\ 2020, Astronomy and Astrophysics, 643, A170
\end{entry}

% \noindent The list of publications has been reproduced with the permission of the copyright holders.